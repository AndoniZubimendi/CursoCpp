\documentclass{beamer}
\special{landscape}

%\usetheme{Berlin}
\usetheme{Warsaw}

%\usecolortheme{seahorse}
%\usefonttheme[onlysmall]{structurebold}

\setbeamertemplate{headline}[split]
\setbeamertemplate{footline}[default]
\setbeamertemplate{footline}[miniframes theme]
%\logo{\includegraphics[scale=0.25]{lifia_logo.png}}

\mode<presentation>
\usepackage[spanish]{babel}
\usepackage{beamerthemesplit}
\usepackage[utf8]{inputenc}
\usepackage{color}      % use if color is used in text

% Comandos en modo Verbatim
%\usepackage{fancyvrb}

\title{Curso C++ - Clase 3}
\author{Juan Antonio Zubimendi\\azubimendi@lifia.info.unlp.edu.ar}

\institute{LIFIA}
%\date{24/04/2008}

\AtBeginSection[]

\begin{document}
 

\begin{frame}
%\frametitle{Presentación}
\titlepage
\end{frame}

\section{Memoria dinámica}
\begin{frame}
\frametitle{Memoria dinámica}
Existen dos tipos de memoria de dinámica disponibles en C++:
\begin{itemize}
 \item \emph{Stack}: Esta memoria se utiliza para las variables locales a las funciones y métodos. Este memoria será devuelta cuando la función finalice.
 \item \emph{Heap}: Con esto podemos obtener memoria para variables que necesitemos más allá del alcance de una función.
\end{itemize}

\end{frame}

\subsection{Heap}
\begin{frame}
\frametitle{Heap}
Para obtener y liberar memoria dinámica (de la Heap) tenemos dos instrucciones:
\begin{itemize}
 \item \emph{new}: Con esta instrucción pedimos memoria dinámica al sistema.
 \item \emph{delete}: Con esta instrucción devolvemos al sistema la memoria asignada a una variable
\end{itemize}

Podemos obtener memoria para:
\begin{itemize}
 \item Para crear objetos
 \item Para tipos básicos u otro tipo de datos.
\end{itemize}

\begin{block}{IMPORTANTE}
   La administración de la memoria pedida con los operadores \emph{new} y \emph{delete} es total responsabilidad del programador.
\end{block}

\end{frame}

\subsection{new}
\begin{frame}[fragile]
\frametitle{new}
Con el operador \emph{new} podemos pedir memoria para un objeto o tipo de datos.

El operador nos devuelve un puntero a la entidad recién construida. De ser necesario se pueden pasar los parámetros necesarios para el constructor.

\begin{verbatim}
   Casa *pCasa = new Casa;
\end{verbatim}

\end{frame}

\subsection{delete}
\begin{frame}
\frametitle{delete}

Con el operador \emph{delete} liberamos la memoria apuntada por el puntero que recibe como parámetro.
El destructor correspondiente a ese objeto es llamado y la memoria liberada.

\end{frame}

\subsection{Punteros}
\begin{frame}[fragile]
\frametitle{Punteros}

Un puntero nos permite un acceso indirecto a un objeto al cual esta apuntando.
Dependiendo al tipo de objeto que apunte, será el tipo de puntero.
Para definir un puntero usamos $*$ antes del nombre de la variable:

\begin{verbatim}
   int * pEntero = new int;
\end{verbatim}

¿Cómo accedemos al valor ``apuntado''?

El acceso al valor apuntado se llama desreferenciar y se puede hacer con el operador $*$:

\begin{verbatim}
   *pEntero = 5;
\end{verbatim}

\end{frame}


\begin{frame}[fragile]
\frametitle{Punteros - continuación}

\begin{verbatim}
    class Punto {
      public:
        Punto(int x, int y);
        int getX();
        int getY();
      private:
        sint _x, _y;
    };

    Punto * p = new Punto(10,10);
\end{verbatim}
\end{frame}



\begin{frame}[fragile]
\frametitle{Punteros - continuación}

Como puedo invocar al metodo \emph{getX} de \emph{p} ?

\begin{verbatim}
    std::cout "Pos X: " << (*p).getX() << std::endl;
\end{verbatim}

Notar el parentesis, para que el operador ``*'' se aplique antes que el operador ``.''.


Como el acceso a miembros de una estructura o clase a partir de un puntero es una operación común, tenemos una manera abreviada de hacerlo:

\begin{verbatim}
    std::cout "Pos X: " << p->getX() << std::endl;
\end{verbatim}

\end{frame}


\begin{frame}[fragile]
\frametitle{Punteros - continuación}

Existe un valor especial que se le puede asignar a un puntero, de manera tal que cual intento de desreferenciarlo nos produzca un error.

Este valor es \emph{0}, a este valor se lo suele llamar puntero nulo (NULL POINTER). En nuestro programa podemos usar la constante NULL como
equivalente del puntero nulo.

\begin{block}{}
Cuando terminamos de usar un puntero o cuando todavía no tiene un valor definido, es buena costumbre asignarle el valor 0 a un puntero.
De esta manera podemos saber si tiene un valor valido o no.
\end{block}

\end{frame}

\begin{frame}[fragile]
\frametitle{Problemas con Punteros}
Los punteros suelen ser la fuente de los problemas más comunes de programación en C++:

\begin{itemize}
 \item Acceder a un puntero no inicializado o no valido (memoria ya devuelta al sistema).
 \item Perder o cambiarle el valor a un puntero que ya tiene un valor asignado.
\end{itemize}

El acceso a un puntero no valido puede provocar un comportamiento indifinido muy díficil de encontrar, inicializando correctamente los punteros y cargando el valor de puntero nulo si no tiene un valor actual es buena practica.

El perder el acceso a una variable dinámica nos produce una perdida de memoria (leak).

\end{frame}

\subsection{new [] y delete []}
\begin{frame}[fragile]
\frametitle{new [] y delete []}

El operador emph{new[]} nos permite pedirle al sistema opertivo más de un elemento a la vez.
Es muy importante utilizar el operador \emph{delete[]}, en lugar de \emph{delete} para liberar la memoria.
\begin{verbatim}
   int *enteros = new int[10];
   for (int i = 0; i < 10; i++)
      enteros[i] = i;
     ...
   delete [] enteros;
\end{verbatim}

¿Alguien nota algo raro?
\end{frame}

\subsection{Punteros y arreglos}
\begin{frame}[fragile]
En el ejemplo anterior utilizamos al puntero como si fuera un arreglo.
En C++, es común utilizar este tipo de notación.
Esta es la manera alternativa de representarlo:
\begin{verbatim}
   int *enteros = new int[10];
   for (int i = 0; i < 10; i++)
      *(enteros + i) = i;
     ...
   delete [] enteros;
 
\end{verbatim}

\end{frame}


\subsection{Leaks}
\begin{frame}
\frametitle{Leaks}

Las perdidas de memoria se producen cuando no liberamos toda la memoria que hemos pedido al sistema operativo. Las causas mas comunes son:
\begin{itemize}
 \item No hacer \emph{delete} de todos los objetos alocados.
 \item Utilizar \emph{delete} en lugar de \emph{delete[]}.
 \item Pedir memoria dentro de una función o método y olvidarnos de liberar la memoria.
 \item Olvidarnos de liberar toda la memoria pedida en un destructor de una clase.
\end{itemize}

\end{frame}

\subsection{operador this}
\begin{frame}
\frametitle{operador this}
  Este operador es valido solamente dentro de cualquier método de instancia de cualquier clase. \\ 
  Apunta a la instancia (es un puntero) específica del objeto que esta ejecutando el método. \\
  Puede ser usado como cualquier otra variable de instancia.

\end{frame}

\section{Herencia de clases}
\subsection{Subclases}
\begin{frame}[fragile]
\frametitle{Subclases}

Para definir una subclase lo hacemos de la siguiente manera:

\begin{verbatim}

 class Cuadrado : public Figura {
   public:
        Cuadrado(); 
 };
\end{verbatim}

De esta manera \emph{Cuadrado} se define como subclase de \emph{Figura}, teniendo acceso a 
metodos y variables de instancia que hereda de la siguiente manera:
\begin{itemize}
 \item Los campos \emph{protected} de la clase base serán accesibles a los metodos de la clase
 \item Los campos \emph{public} de la clase base serán accesibles desde los metodos y a los usuarios de los objetos.
 \item Los campos \emph{private} de la clase, no serán accesible ni a los metodos ni a los usuarios de los objetos.
\end{itemize}
\end{frame}

\begin{frame}[fragile]
\frametitle{Herencia multiple}
C++ nos ofrece herencia multiple para los objetos. 
Si bien no es muy común y en la mayoría de los casos recomendamos no usarla.

\begin{verbatim}
 class Avion : public Transporte,
               public BuscarMejorEjemplo {
   public:
      Avion();
 };
\end{verbatim}

\end{frame}

\subsection{Punteros y Polimorfismo}
\begin{frame}
\frametitle{Punteros y Polimorfismo}
C++ nos deja aprovechar el carácter polimorfico de los objetos a través de los punteros.
A un puntero de una clase base puede apuntar a un objeto de esa clase o de cualquier subclase.

\end{frame}

\begin{frame}[fragile]
\frametitle{Punteros y Polimorfismo - Ejemplo}
\begin{verbatim}
#include <iostream>
class A {
public:
    void imprime() {
        std::cout << "hola, soy A" << std::endl;
    }
};

class B : public A {
public:
   void imprime() {
      std::cout << "hola, soy B" << std::endl;
   }
};

\end{verbatim}

\end{frame}

\begin{frame}[fragile]
\frametitle{Punteros y Polimorfismo - Ejemplo}
\begin{verbatim}

void imprimir(A *objeto) {
   std::cout << "objeto metodo imprime()" << std::endl;
   objeto->imprime();
} 
int main(int argc, char **argv) {
   A *objetoA = new A;
   B *objetoB = new B;
   imprimir(objetoA);
   imprimir(objetoB);

   delete objetoA; delete objetoB;
   return 0;
}
\end{verbatim}

\end{frame}

\subsection{Métodos virtuales}
\begin{frame}
\frametitle{Métodos virtuales}
¿Qué pasó? \\

Por cuestiones de eficiencia, las busquedas de los métodos polimorficos en C++ no se buscan en las subclases.
Para indicarle al compilador de C++ que un método es polimorfico debemos anteponer al mismo la palabra clave ``virtual''.
En C++ llamamos a estos métodos virtuales.

\end{frame}

\begin{frame}[fragile]
\frametitle{Punteros y Polimorfismo - Ejemplo Corregido}
\begin{verbatim}
#include <iostream>
class A {
public:
    virtual void imprime() {
        std::cout << "hola, soy A" << std::endl;
    }
};

class B : public A {
public:
   void imprime() {
      std::cout << "hola, soy B" << std::endl;
   }
};

\end{verbatim}

\end{frame}


\subsection{Clases abstractas}
\begin{frame}[fragile]
\frametitle{Clases abstractas}

Cuando diseñamos clases abstractas, podemos querer dejar la implementación de una función a las subclases.
Podemos hacerlo igualando a cero el método que queremos marcar como abstracto.

\begin{verbatim}
 
  class Figura {
public:
     Figura();

     virtual int area() = 0;
  };
\end{verbatim}

Obviamente, no podremos instanciar un objeto de una clase que contenga por lo menos un método abstracto.
\end{frame}

\subsection{Interfaces}
\begin{frame}
\frametitle{Interfaces}
Si bien C++ no nos ofrece interfaces como Java, podemos lograr un resultado similar haciendo:
\begin{itemize}
  \item Definimos la interface como una clase abstracta con todos los mensajes que queremos que tenga la interfaz. Cada uno de estos métodos los indicamos como abstractos.
  \item A la clase que queremos que tenga esa interfaz, la hacemos heredar a partir de la clase creada en el paso anterior.
\end{itemize}
\end{frame}

\end{document}

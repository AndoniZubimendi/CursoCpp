\documentclass{beamer}
\special{landscape}

%\usetheme{Berlin}
\usetheme{Warsaw}

%\usecolortheme{seahorse}
%\usefonttheme[onlysmall]{structurebold}

\setbeamertemplate{headline}[split]
\setbeamertemplate{footline}[default]
\setbeamertemplate{footline}[miniframes theme]
%\logo{\includegraphics[scale=0.25]{lifia_logo.png}}

\mode<presentation>
\usepackage[spanish]{babel}
\usepackage{beamerthemesplit}
\usepackage[utf8]{inputenc}
\usepackage{color}      % use if color is used in text


% Comandos en modo Verbatim
%\usepackage{fancyvrb}


\title{Curso C++ - Clase 2}
\author{Juan Antonio Zubimendi\\azubimendi@lifia.info.unlp.edu.ar}

\institute{LIFIA}
%\date{24/04/2008}

\AtBeginSection[]

\begin{document}
 
\begin{frame}
\frametitle{}
\titlepage
\end{frame}


\section{Otros tipos de Datos}

\subsection{Arreglos}
\begin{frame}[fragile]
\frametitle{Arreglos}
Los arreglos se definen de la siguiente manera:

\begin{verbatim}
  <tipo> nombre_arreglo[<tamaño>];
\end{verbatim}

Se puede inicializar el arreglo si se desea.
El indice del primer elemento es 0.
El \emph{tamaño} puede ser ser un valor constante, o una expresión entera no negativa.
\begin{verbatim}
  int n = 5;
  char caracteres[n];
\end{verbatim}

\end{frame}

\subsection{typedef}
\begin{frame}[fragile]
\frametitle{typedef}
Define un alias de tipo. No se define un tipo de datos nuevo, sino otro nombre para el mismo tipo de datos.

\begin{verbatim}
 typedef tipoExistente tipoNuevo;
\end{verbatim}

\end{frame}

\subsection{enum}
\begin{frame}[fragile]
\frametitle{enum}
Define un tipo de datos enumerado.
\begin{verbatim}
 enum <nombre> {
    PRIMERO,
    SEGUNDO = 1,
    TERCERO,
    CUARTO
 };
\end{verbatim}

El tipo será un alias de \emph{int}, los valores seran autoincrementales. Si no se asigna ningún valor inicial, el primer valor será 0.
\end{frame}

\subsection{struct}
\begin{frame}[fragile]
\frametitle{struct}
Definen tipos de datos propios. Definen un conjunto de campos y/o funciones propias.

\begin{verbatim}
struct Cuadrado {
  int lado;
  int area() { return lado * lado; }
};
\end{verbatim}

Puedo acceder a los miembros de una estructura:
\begin{verbatim}
  Cuadrado c;
  c.lado = 5;
\end{verbatim}

\end{frame}


\section{Clases}
\begin{frame}[fragile]
\frametitle{Clases}

Para definir una clase usamos:
\begin{verbatim}
  class Punto {
    private:
      int x, y;
     public:
      int getX() { return x;}
      int getY() { return y;}
  };
\end{verbatim}

\end{frame}

\subsection{Visibilidad}
\begin{frame}
\frametitle{Visibilidad}
Cada miembro de una clase puede tener una de estas tres visibilidades:
\begin{itemize}
 \item Public: El acceso a este miembro de la estructura es irrestricto.
 \item Private: Solo se puede acceder dentro de la misma clase.
 \item Protected: La clase y sus subclases pueden acceder a este miembro.
\end{itemize}

Dentro de la declaración de la clase se puede definir la visibilidad escribiendo
 la misma seguida por dos puntos para indicar que de ahí en adelante todo lo que se declare tendrá esa visibilidad.
Si no se indica nada la visibilidad será \emph{Private}.
\begin{block}{}
 \emph{struct} y \emph{class} son totalmente equivalentes, la única diferencia es la visibilidad que ofrecen por omisión.
\end{block}

\end{frame}

\subsection{Métodos}
\begin{frame}[fragile]
\frametitle{Métodos}
   Los métodos se pueden definir en la misma declaración de la clase. \\
   Pero es preferible definir una función fuera de la declaración de la clase. 
   \begin{verbatim}
    class Cuadrado {
        Cuadrado(int lado);
      private:
        int _lado;
      public:
        int area();
    };
\end{verbatim}
\end{frame}

\begin{frame}[fragile]
\frametitle{Métodos}
   
   Una vez definida la clase, hacemos:
\begin{verbatim}
  int Cuadrado::Cuadrado(int lado) {
       _lado = lado;
  }

  int Cuadrado::area() {
      return lado * lado;
  }
\end{verbatim}

\begin{block}
Es conveniente separar la declaración y la definición en archivos diferentes.
\end{block}
\end{frame}

\subsection{Constructores}

\begin{frame}[fragile]
\frametitle{Constructores}

Cada clase posee por lo menos un constructor.

Si no indicamos uno especificamente, se creara uno sin parámetros.

Los constructores son funciones de una clase con el mismo nombre de la clase.

Podemos tener más de un constructor, es decir podemos sobrecargarlo.

Creamos instancias de un objeto invocando a uno de los constructores de la clase.

En el constructor podemos reservar e inicializar los aspectos del objeto que sean necesarios.

\end{frame}

\begin{frame}[fragile]
\frametitle{Ejemplo}
\begin{verbatim}
 
  class Figura {
    public:
      Figura(int lados);
      Figura();

      int lados() const;
  };
\end{verbatim}

\end{frame}

%\begin{frame}
%\frametitle{Constructores importantes}
%
%\begin{itemize}
% \item Constructor de copia
%\end{itemize}
%
%\end{frame}

\subsection{Destructores}
\begin{frame}
\frametitle{Destructores}

Cuando un objeto va a ser destruido se invoca al destructor de la clase.

El destructor es uno solo, a diferencia de los constructores. 

El destructor no toma parámetros ni devuelve parámetros.

El nombre del método será un $~$ seguido del nombre de la clase.

\begin{block}
Acordarse de liberar recursos utilizados por el objeto.
\end{block}

\end{frame}

\subsection{Variables y Métodos de Clase}
\begin{frame}[fragile]
\frametitle{Variables y Métodos de Clase}
   Los métodos y variables de clase, se definen de la misma manera que los de instancia, la única diferencia
   es que se antepone la palabra clave \emph{static} antes de la declaración del método o variable. \\
\end{frame}

\begin{frame}[fragile]
\begin{verbatim}
  class Figura {
    public:
        Figura();
        static int figurasTotales();

    private:
      static int _totalFiguras;

  };

  Figura::_totalFiguras = 0;
  int figurasTotales() {
      return _totalFiguras;
  }
\end{verbatim}

\end{frame}

\section{Operador const}
\begin{frame}
\frametitle{Uso de const}
  \emph{const} afecta de diferentes maneras a las expresiones donde aparece.

\begin{itemize}
 \item Como modificador - antes - de una variable, es una constante.
 \item Como modificador - antes - de un párametro, que ese parámetro no se puede modificar.

 \item Al final la declaración de un metodo, que dicho método no modifica el estado del objeto. Esto le permite al compilador realizar optimizaciones.
\end{itemize}
\end{frame}

\subsection{Referencias}
\begin{frame}[fragile]
\frametitle{Referencias}
  
 Una referencia es un alias de una variable.

 Definimos una referencia de la siguiente manera:

\begin{verbatim}
   int entero;

   int & referencia = entero;
\end{verbatim}

Una vez asociada una referencia con una variable, no se puede cambiar. En el momento de definir la referencia hay que asociarla a la variable a la que hacemos referencia.

Dentro de una función o método podemos retornar una referencia a una variable.


\begin{block}
 OJO! Con las referencias a las variables locales
\end{block}

\end{frame}

\subsection{Referencias y el pasaje de parámetros}
\begin{frame}[fragile]
\frametitle{Referencias}

Hasta ahora habiamos visto que todas las pasajes de parámetros a funciones o métodos eran por valor, ahora vemos que también podemos pasar parámetros por referencia:

\begin{verbatim}
   Matriz multiplicar(Matriz &m, Matriz &n);
\end{verbatim}
Si no pasaramos por referencia, los objetos 
\end{frame}

\section{Operador static}

\begin{frame}
\frametitle{Uso de static}
  El modificador \emph{static} se puede utilizar

\begin{itemize}
 \item Como modificador - antes - de una variable global o función, esta función o variable no se podrá utilizar fuera del archivo donde este definido.
 \item Como modificador - antes - de una variable local a una función o método, esta variable no se destruirá cuando la función termine, persistiendo su valor a través de las diferentes llamadas a la misma.
\end{itemize}
\end{frame}

\section{Trabajando con muchos archivos}

\begin{frame}
\frametitle{Trabajando con multiples archivos}
Es conveniente
\begin{itemize}
 \item Separar cada objeto en un archivo distinto, para 
 \item Separar la implementación de la interfaces de los objetos.
\end{itemize}

\end{frame}

\subsection{Trabajando con muchos archivos}

\begin{frame}
\frametitle{Separar implementación de interfaz}
Sabiendo que C++ separa claramente la definición de la declaración de tipos de datos y de funciones, podemos usar eso para realizar la separación.
\begin{itemize}
 \item En archivos de encabezados ``\emph{headers}'': Pondremos las definiciones. Estos archivos suelen llevar la extensión \emph{.h}.
 \item En archivos de implementación escribiremos las definiciones, pero incluyendo previamente las declaraciones necesarias.
 \item Si existe una dependencia de una clase con otra, podremos incluir el respectivo archivo de encabezado para que el compilador pueda acceder correctamente a todas las clases.
\end{itemize}
\end{frame}


\subsection{Archivos de Encabezados}
\begin{frame}
\frametitle{Archivo de encabezados}
\begin{itemize}
 \item Estos archivos suelen tener la extensión \emph{.h} y llevan el mismo nombre que el archivo de implementación.
 \item Aqui escribiremos todas las definiciones que necesitemos usar del archivo de implementación.
 \item Los archivos de encabezado se incluyen con la directiva de preprocesador \emph{\#incldue}.
\end{itemize}

\end{frame}

\subsection{Guardas en los encabezados}
\begin{frame}[fragile]
\frametitle{Guardas en los archivos de encabezados}
a.h:
\begin{verbatim}
#include "b.h"

struct A
{
  struct B *b;
} A;
\end{verbatim}

b.h:
\begin{verbatim}
#include "a.h"

typedef struct B
{
  struct A *a;
} B;
\end{verbatim}

¿Qué pasa aca?
\end{frame}

\begin{frame}[fragile]
\frametitle{Guardas en los archivos de encabezados}
Para evitar esto, usamos guardas sobre los archivos de encabezados.

\begin{verbatim}
#ifndef __A_H__
#define __A_H__

#include "b.h"

struct A
{
  struct B *b;
} A;

#endif
\end{verbatim}

b.h:
\begin{verbatim}
#ifndef __B_H__
#define __B_H__

#include "a.h"

typedef struct B
{
  struct A *a;
} B;

#endif
\end{verbatim}


\end{frame}

\subsection{Archivos de Implementación}
\begin{frame}
 \frametitle{Archivo de implementación}
\begin{itemize}

 \item Estos archivos suelen tener la extensión \emph{.cpp}, o \emph{.cxx}
 \item Este archivo incluye (mediante \emph{\#include}) el archivo de encabezado propio y los demas archivos de encabezados que necesite.
\end{itemize}

\end{frame}

\section{Testing - Google Test}

\subsection{Introducción}
\begin{frame}
\frametitle{Google Test}
Vamos a ver como utilizar una librería de Test de Unidad de C++. \\
¿Por qué Google Test?
\begin{itemize}
 \item Porque está basada en la arquitectura xUnit (sUnit, jUnit...)
 \item Porque es fácil de usar
 \item Porque es software libre
\end{itemize}

La página web del sitio está en http://code.google.com/p/googletest/

\end{frame}

\subsection{Instalación}
\begin{frame}
\frametitle{Instalación de Google Test}
Los pasos son:
\begin{itemize}
 \item Descargar la librería
 \item Descomprimirla, ejecutando: \emph{tar xjf gtest-1.5.2.tar.bz2}
 \item Compilarla, ejecutando: \emph{make}
 \item Instalarla, ejecutando: \emph{make install}
\end{itemize}

\end{frame}

\subsection{Uso}
\begin{frame}
\frametitle{Instalación de Google Test}
En el curso le proveemos un ejemplo completo basado en el tutorial. \\
La Wiki del proyecto posee una buena introducción:
  http://code.google.com/p/googletest/wiki/Primer
\end{frame}

\end{document}

